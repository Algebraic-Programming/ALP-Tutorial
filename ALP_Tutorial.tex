
\section{Installation on Linux}\label{sec:installation}

This section explains how to install ALP on a Linux system and compile a simple example. To get started:

\begin{enumerate}
\item \textbf{Install prerequisites}. Ensure you have a C++11 compatible compiler (e.g. \texttt{g++} 4.8.2 or later) with OpenMP support, CMake (>= 3.13) and GNU Make, plus development headers for libNUMA and POSIX threads. 
For example, on Debian/Ubuntu:
\begin{verbatim}
sudo apt-get install build-essential libnuma-dev libpthread-stubs0-dev cmake;
\end{verbatim}
or, on Red Hat systems (as root):
\begin{verbatim}
dnf group install "Development Tools"
dnf install numactl-devel cmake
\end{verbatim}

\item \textbf{Obtain ALP}. Download or clone the ALP repository, e.g. from its official GitHub location:
\begin{verbatim}
git clone https://github.com/Algebraic-Programming/ALP.git
\end{verbatim}

\item \textbf{Build ALP}. Create a build directory and invoke the provided bootstrap script to configure the project with CMake, then compile and install:
\begin{lstlisting}[language=bash]
$ cd ALP && mkdir build && cd build
$ ../bootstrap.sh --prefix=../install # configure the build
$ make -j # compile the ALP library
$ make install # install to ../install
\end{lstlisting}
(You may choose a different installation prefix as desired.)

\item \textbf{Set up environment}. After installation, activate the ALP environment by sourcing the script setenv in the install directory:
\begin{lstlisting}[language=bash]
$ source ../install/bin/setenv
\end{lstlisting}
This script updates the shell PATH to make the ALP compiler wrapper accessible, as well as adds the ALP libraries to the applicable standard library paths.

\item \textbf{Compile an example}. ALP provides a compiler wrapper \texttt{grbcxx} to compile programs that use the ALP/GraphBLAS API. This wrapper automatically adds the correct include paths and links against the ALP library and its dependencies. For example, to compile the provided sp.cpp sample:
\begin{lstlisting}[language=bash]
$ grbcxx ../examples/sp.cpp -o sp_example
\end{lstlisting}
By default this produces a sequential program; you can add the option \texttt{-b reference\_omp} to use the OpenMP parallel backend for shared-memory auto-parallelisation. The wrapper \texttt{grbcxx} accepts other backends as well (e.g.\ \texttt{-b nonblocking} for nonblocking execution on shared-memory parallel systems or \texttt{-b hybrid} for hybrid shared- and distributed-memory execution\footnote{The \texttt{hybrid} backend requires the Lightweight Parallel Foundations, LPF, is installed as an additional dependence; see \url{https://github.com/Algebraic-Programming/LPF/} and/or the main ALP \texttt{README.md}.}).

\item \textbf{Run the program}. Use the provided runner \texttt{grbrun} to execute the compiled binary. For a simple shared-memory program, running with \texttt{grbrun} is similar to using \texttt{./program} directly. For example:
\begin{lstlisting}[language=bash]
$ grbrun ./sp_example
\end{lstlisting}
(The \texttt{grbrun} tool is more relevant when using distributed backends or controlling the execution environment; for basic usage, the program can also be run directly.)
\end{enumerate}
After these steps, you have installed ALP and have made sure its basic functionalities are operational. In the next sections we introduce core ALP/GraphBLAS concepts and walk through a simple example program.

\section{ALP/GraphBLAS}\label{sec:alp_concepts}

ALP exposes a GraphBLAS interface which separate in three categories: 1) algebraic containers (vectors, matrices, etc.); 2) algebraic structures (binary operators, semrings, etc.); and 3) algebraic operations that take containers and algebraic structures as arguments. This interface was developed in tandem with what became the GraphBLAS C specification, however, is pure C++. All containers, primitives, and algebraic structures are defined in the \texttt{grb} namespace. The ALP user documentation may be useful in course of the exercises. These may be found at: \url{http://albert-jan.yzelman.net/alp/user/}.

Let us first bootstrap our tutorial with a simple \emph{Hello World} example:

\begin{lstlisting}[language=c++,morekeywords=constexpr,morekeywords=size_t]
#include <cstddef>
#include <cstring>

#include <graphblas.hpp>

#include <assert.h>

constexpr size_t max_fn_size = 255;
typedef char Filename[ max_fn_size ];

void hello_world( const Filename &in, int &out ) {
	std::cout << "Hello from " << in << std::endl;
	out = 0;
}

int main( int argc, char ** argv ) {
	// get input
	Filename fn;
	(void) std::strncpy( fn, argv[ 0 ], max_fn_size );

	// set up output field
	int error_code = 100;

	// launch hello world program
	grb::Launcher< grb::AUTOMATIC > launcher;
	assert( launcher.exec( &hello_world, fn, error_code, true )
		== grb::SUCCESS );

	// return with the hello_world error code
	return error_code;
}
\end{lstlisting}

In this code, we have a very simple \texttt{hello\_world} function that takes its own filename as an input argument, prints a hello statement to \texttt{stdout}, and then returns a zero error code.
ALP uses the concept of a \emph{Launcher} to start ALP programs such as \texttt{hello\_world}, which is examplified in the main function above. This mechanism allows for encapsulation and starting sequences of ALP programs, potentially adaptively based on run-time conditions. The signature of an ALP program always consists of two arguments: the first being program input and the second being program output. The types of both input and output may be any POD type.

Assuming the above is saved as \texttt{alp\_hw.cpp}, it may be compiled and run as follows:
\begin{lstlisting}[language=bash]
$ grbcxx alp_hw.cpp
$ grbrun ./a.out
Info: grb::init (reference) called.
Hello from ./a.out
Info: grb::finalize (reference) called.
$ 
\end{lstlisting}

\noindent \textbf{Exercise 1.} Double-check that you have the expected output from this example, as we will use its framework in the following exercises.

\noindent \textbf{Question.} Why is \texttt{argv[0]} not directly passed as input to \texttt{hello\_world}?

\noindent \textbf{Bonus Question.} Consider the \href{http://albert-jan.yzelman.net/alp/user/classgrb_1_1Launcher.html#af33a2d0ff876594143988613ebaebae7}{programmer reference documentation for the \texttt{grb::Launcher}}, and consider distributed-memory parallel execution in particular. Why is the last argument to \texttt{launcher.exec} \texttt{true}?


\subsection{ALP/GraphBLAS Containers}

The primary ALP/GraphBLAS container types are \texttt{grb::Vector<T>} and \texttt{grb::Matrix<T>}. These are templated on a value type \texttt{T}, the type of elements stored. The type \texttt{T} can be any plain-old-data (POD) type, including \texttt{std::pair} or \texttt{std::complex<T>}. Both vectors and matrices can be sparse, meaning they efficiently represent mostly-zero data by storing only nonzero elements. For example, one can declare a vector of length $100\ 000$ and a $150\ 000\times100\ 000$ matrix as:
\begin{lstlisting}
grb::Vector<double> x(100000), y(150000);
grb::Matrix<void> A(150000, 100000);
\end{lstlisting}
In this snippet, \texttt{x} and \texttt{y} are vectors of type \texttt{double}. The matrix \texttt{A} is declared with type \texttt{void}, which signifies it only holds the pattern of nonzero positions and no numeric values. Perhaps more commonly, one would use a numeric type (e.g. \texttt{double}) for holding matrix nonzeroes. A \texttt{void} matrix as in the above example is useful for cases where existence of an entry is all that matters, as e.g.\ for storing Boolean matrices or unweighted graphs.

By default, newly instantiated vectors or matrices are empty, meaning they store no elements. You can query properties like length or dimensions via \texttt{grb::size(vector)} for vector length or \texttt{grb::nrows(matrix)} and \texttt{grb::ncols(matrix)} for matrix dimensions. The number of elements present within a container may be retrieved via \texttt{grb::nnz(container)}. Containers have a maximum capacity on the number of elements they may store; the capacity may be retrieved via \texttt{grb::capacity(container)} and on construction of a container is set to the maximum of its dimensions. For example, the initial capacity of \texttt{x} in the above is $100\ 000$, while that of \texttt{A} is $150\ 000$. The size of a container once initialised is fixed, while the capacity may increase during the lifetime of a container.

\noindent \textbf{Exercise 2.} Allocate vectors and matrices in ALP as follows:
\begin{itemize}
  \item a \texttt{grb::Vector<double>} \texttt{x} of length 100, with initial capacity 100;
  \item a \texttt{grb::Vector<double>} \texttt{y} of length 1\ 000, with initial capacity 1\ 000;
  \item a \texttt{grb::Matrix<double>} \texttt{A} of size $(100 \times 1\ 000)$, with initial capacity 1\ 000; and
  \item a \texttt{grb::Matrix<double>} \texttt{A} of size $(100 \times 1\ 000)$, with initial capacity 5\ 000.
\end{itemize}
You may start from a copy of \texttt{alp\_hw.cpp}. Employ \texttt{grb::capacity} to print out the capacities of each of the containers. \textbf{Hint:} refer to the user documentation on how to override the default capacities.

If done correctly, you should observe output similar to:

\begin{lstlisting} [language=bash, basicstyle=\ttfamily\small, showstringspaces=false]
Info: grb::init (reference) called.
Capacity of x: 100
Capacity of y: 1000
Capacity of A: 1000
Capacity of B: 5000
Info: grb::finalize (reference) called.
\end{lstlisting}

\noindent \textbf{Question.} Is overriding the default capacity necessary for all of \texttt{x, y, A} in the above exercise?

\subsection{Basic Container I/O}

The containers in the C++ Standard Template Library (STL) employ the concept of iterators to ingest and extract data, as well as foresees in primitives for manipulating container contents.
ALP/GraphBLAS is no different, e.g., providing \texttt{grb::clear(container)} to remove all elements from a container, similar to the clear function defined by STL vectors, sets, et cetera.
Similarly, \texttt{grb::set(vector,scalar)} sets all elements of a given vector equal to the given scalar, resulting in a full (dense) vector.
By contrast, \texttt{grb::setElement(vector,scalar,index)} sets only a given element at a given index equal to a given scalar.

\noindent \textbf{Exercise 3.} Start from a copy of \texttt{alp\_hw.cpp} and modify the \texttt{hello\_world} function to allocate two vectors and a matrix as follows:
\begin{itemize}
	\item a \texttt{grb::Vector<bool>} \texttt{x} and \texttt{y} both of length $497$ with capacities $497$ and $1$, respectively;
	\item a \texttt{grb::Matrix<void>} \texttt{A} of size $497\times497$ and capacity $1\ 727$.
\end{itemize}
Then, initialise $y$ with a single value \texttt{true} at index $200$, and initialise $x$ with \texttt{false} everywhere. Print the number of nonzeroes in $x$ and $y$. Once done, after compilation and execution, the output should be alike:
\begin{lstlisting}
...
nonzeroes in x: 497
nonzeroes in y: 1
...
\end{lstlisting}

\noindent \textbf{Bonus question.} Print the capacity of $y$. Should the value returned be unexpected, considering the specification in the user documentation, is this a bug in ALP?

ALP/GraphBLAS containers are compatible with standard STL output iterators. For example, the following for-loop prints all entries of $y$:
\begin{lstlisting}
for( const auto &pair : y ) {
	std::cout << "y[ " << pair.first << " ] = " << pair.second << "\n";
}
\end{lstlisting}

\noindent \textbf{Exercise 4.} Use output iterators to double-check that $x$ has $497$ values and that all those values equal \texttt{false}.

Commonly, matrices are available in common file exchange formats, such as MatrixMarket \texttt{.mtx}. To facilitate working with standard files, ALP contains utilities for reading standard format. The utilities are not included with \texttt{graphblas.hpp} and must instead be included explicitly:
\begin{lstlisting}
#include <graphblas/utils/parser.hpp>
\end{lstlisting}
Including the above parser utility defines the \texttt{MatrixFileReader} class. Its constructor takes one filename plus a Boolean that describes whether vertex are numbered consecutively (as required in the case of MatrixMarket files); some graph repositories, e.g. SNAP, have non consecutively-numbered vertices which could be an artifact of how the data is constructed or due to post-processing. In this case, passing \texttt{false} as the second argument to the parser will map the non-consecutive vertex IDs to a consecutive range instead, thus packing the graph structure in a minimally-sized sparse matrix. In this tutorial, however, we stick to MatrixMarket files and therefore always pass \texttt{true}:
\begin{lstlisting}
grb::utils::MatrixFileReader< double > parser( in, true );
\end{lstlisting}
After instantiation, the parser defines STL-compatible iterators that are enriched for use with sparse matrices; e.g., one may issue
\begin{lstlisting}
const auto iterator = parser.begin();
std::cout << "First parsed entry: ( " << iterator.i() << ", " << iterator.j() << " ) = " << iterator.v() << "\n";
\end{lstlisting}
which should print, on execution,
\begin{lstlisting}
First parsed entry: ( 495, 496 ) = 0.897354
\end{lstlisting}
Note that the template argument to \texttt{MatrixFileReader} defines the value type of the sparse matrix nonzero values. The start-end iterator pair from this parser is compatible with the \texttt{grb::buildMatrixUnique} ALP/GraphBLAS primitive, where the suffix -unique indicates that the iterator pair should never iterate over a nonzero at the same matrix position $(i,j)$ more than once. Hence reading in the matrix into the ALP/GraphBLAS container $A$ proceeds simply as
\begin{lstlisting}
grb::RC rc = grb::buildMatrixUnique(
    A,
    parser.begin( grb::SEQUENTIAL ), parser.end( grb::SEQUENTIAL ),
    grb::SEQUENTIAL
);
assert( rc == grb::SUCCESS );
\end{lstlisting}
The type \texttt{grb::RC} is the standard return type; ALP primitives\footnote{that are not simple `getters' like \texttt{grb::nnz}} always return an error code, and, if no error is encountered, return \texttt{grb::SUCCESS}. Iterators in ALP may be either \emph{sequential} or \emph{parallel}. Start-end iterator pairs that are sequential, such as retrieved from the parser in the above snippet, iterate over all elements of the underlying container (in this case, all nonzeroes in the sparse matrix file). A parallel iterator, by contrast, only retrieves some subset of elements $V_s$, where $s$ is the process ID. It assumes that there are a total of $p$ subsets $V_i$, where $p$ is the total number of processes. These subsets are pairwise disjoint (i.e., $V_i\cap V_j=\emptyset$ for all $i\neq j, 0\leq i,j<p$), while $\cup V_i$ corresponds to all elements in the underlying container. Parallel iterators are useful when launching an ALP/GraphBLAS program with multiple processes to benefit from distributed-memory parallelisation; in such cases, it would be wasteful if every process iterates over all data elements on data ingestion-- instead, parallel I/O is preferred. ALP primitives that take iterator pairs as input must be aware of the I/O type, which is passed as the last argument to \texttt{grb::buildMatrixUnique} in the above code snippet.

\noindent \textbf{Exercise 5.} Use the \texttt{FileMatrixParser} and its iterators to build $A$ from \texttt{west0497.mtx}. Have it print the number of nonzeroes in $A$ after buildMatrixUnique. Then modify the \texttt{main} function to take as the first program argument a path to a .mtx file, pass that path to the ALP/GraphBLAS program. Find and download the west0497 matrix from the SuiteSparse matrix collection, and run the application with the path to the downloaded matrix. If all went well, its output should be something like:
\begin{lstlisting}[keywordstyle=\ttfamily]
Info: grb::init (reference) called.
elements in x: 497
elements in y: 1
y[ 200 ] = 1
Info: MatrixMarket file detected. Header line: ``%%MatrixMarket matrix coordinate real general''
Info: MatrixFileReader constructed for /home/yzelman/Documents/datasets/graphs-and-matrices/west0497.mtx: an 497 times 497 matrix holding 1727 entries. Type is MatrixMarket and the input is general.
First parsed entry: ( 495, 496 ) = 0.897354
nonzeroes in A: 1727
Info: grb::finalize (reference) called.
\end{lstlisting}

\noindent \textbf{Bonus question.} Why is there no \texttt{grb::set(matrix,scalar)} primitive?

\subsection{Copying, Masking, and Standard Matrices}

Continuing from the last exercise, the following code would store a copy of $y$ in $x$ and a copy of $A$ in $B$:
\begin{lstlisting}
grb::Matrix< double > B( 497, 497 );
assert( grb::capacity( B ) == 497 );
grb::RC rc = grb::set( x, y );
rc = rc ? rc : grb::set( B, A, grb::RESIZE );
rc = rc ? rc : grb::set( B, A, grb::EXECUTE );
\end{lstlisting}
\noindent \textbf{Question.} What does the code pattern \texttt{rc = rc ? rc : <function call>;} achieve?

Note that after instantiation of $B$ (line 1 in the above) it will be allocated with a default capacity of $497$ values maximum (line 2). However, $A$ from the preceding exercise has $1\ 727$ values; therefore, simply executing \texttt{grb::set( B, A )} would return \texttt{grb::ILLEGAL}. Rather than manually having to call \texttt{grb::resize( B, 1727 )} to make the copy work, the above snippet instead first requests ALP/GraphBLAS to figure out the required capacity of $B$ and resize it if necessary (line 4), before then executing the copy (line 5). The execute phase is default-- i.e., the last line could equivalently have read \texttt{rc = rc ? rc : grb::set( B, A );}. Similarly, the vector copy (line 3) could have been broken up in resize-execute phases, however, from the previous exercises we already learned that the default vector capacity guarantees are sufficient to store $y$ and so we call the execute phase immediately.

\noindent \textbf{Question.} $A$ contains $1727$ double-precision elements. Are these $1727$ nonzeroes?

The \texttt{grb::set} primitive may also take a mask argument. The mask determines which outputs are computed and which outputs are discarded. For example, recall that $y$ from the previous exercise is a vector of size $497$ that contains only one value $y_{200}=$\texttt{true}. Then
\begin{lstlisting}
grb::RC rc = grb::set( x, y, false );
\end{lstlisting}
results in a vector $x$ that has one entry only: $x_{200}=\texttt{false}$. This is because $y$ has no elements anywhere except at position $200$, and hence the mask evaluates \texttt{false} for any $0\leq i<497, i\neq200$, and no entries are generated by the primitive at those positions. At position $200$, however, the mask $y$ does contain an element whose value is \texttt{true}, and hence the \texttt{grb::set} primitive will generate an output entry there. The value of the entry $x_{200}$ is set to the value given to the primitive, which is \texttt{false}. All GraphBLAS primitives with an output container can take mask arguments.

\noindent \textbf{Question.} What would \texttt{grb::set( y, x, true )} return for $y$, assuming it is computed immediately after the preceding code snippet?

We have shown how matrices may be built from input iterators while similarly, vectors may be built from standard iterators through \texttt{grb::buildVectorUnique} as well. Iterator-based ingestion also allows for the construction of vectors and matrices that have regular structures. ALP/GraphBLAS comes with some standard recipes that exploit this, for example to build an $n\times n$ identity matrix:
\begin{lstlisting}
...
#include <graphblas/algorithms/matrix_factory.hpp>
...

    const grb::Matrix< double > identity = grb::algorithms::matrices< double >::identity( n );

...
\end{lstlisting}
\href{http://albert-jan.yzelman.net/alp/v0.8-preview/classgrb_1_1algorithms_1_1matrices.html#a1336accbaf6a61ebd890bef9da4116fc}{Other regular patterns supported} are \emph{eye} (similar to identity but not required to be a square matrix) and \emph{diag} (which takes an optional parameter $k$ to generate offset the diagonal, e.g., to generate a superdiagonal matrix). The class includes a set of constructors that result in dense matrices as well, including \emph{dense}, \emph{full}, \emph{zeros}, and \emph{ones}; note, however, that ALP/GraphBLAS is not optimised to handle dense matrices efficiently and so their use is discouraged\footnote{for similar reasons, actually, there is no primitive \texttt{grb::set(matrix,scalar)} in the GraphBLAS API}. While constructing matrices from standard file formats and through general iterators hence are key features for usability, it is also possible to derive matrices from existing ones via \texttt{grb::set(matrix,mask,value)}.

\noindent \textbf{Exercise 6.} Copy the code from the previous exercise, and modify it to determine whether $A$ holds explicit zeroes; i.e., entries in $A$ that have numerical value zero. \textbf{Hint:} it is possible to complete this exercise using only masking and copiying. You may also want to change the element type of $A$ to \texttt{double} (or convince yourself that this is not required).

\noindent \textbf{Exercise 7.} Determine how many explicit zeroes exist on the diagonal, superdiagonal, and subdiagonal of $A$; i.e., compute $|\{A_{ij}\in A\ |\ A_{ij}=0, |i-j|\leq1, 0\leq i,j<497\}|$.

\noindent \textbf{Bonus question.} How much memory beyond that which is required to store the $n\times n$ identity matrix will a call to \texttt{matrices< double >::identity( n )} consume? \textbf{Hint:} consider that the iterators passed to \texttt{buildMatrixUnique} iterate over regular index sequences that can easily be systematically enumerated.

\subsection{Numerical Linear Algebra}

GraphBLAS, as the name implies, provides canonical BLAS-like functionalities on sparse matrices, sparse vectors, and dense vectors. These include \texttt{grb::dot}, \texttt{grb::eWiseAdd}, \texttt{grb::eWiseMul}, \texttt{grb::mxv}, \texttt{grb::vxm}, and \texttt{grb::mxm}. The former three scalar/vector primitives are dubbed \emph{level 1}, the following two matrix--vector primitives \emph{level 2}, and the latter matrix--matrix primitive \emph{level 3}. Their functionalities are summarised as follows:\newline

      \textbf{grb::dot} – Compute the dot product of two vectors, $\alpha$\textit{+=}$u^Tv$ or $\alpha$\textit{+=}$\sum_i (u_i \times v_i)$, in essence combining element-wise multiplication with a reduction. The output $\alpha$ is a scalar, usually a primitive type such as \texttt{double}. Unlike the out-of-place \texttt{grb::set}, the \texttt{grb::dot} updates the output scalar in-place.\newline

	\textbf{grb::eWiseMul, grb::eWiseAdd} - These primitives combine two containers element-by-element, the former using element-wise multiplication, and the latter using element-wise addition. Different from \texttt{grb::set}, these primitives are in-place, meaning the result of the element-wise operations are added to any elements already in the output container; i.e., \texttt{grb::eWiseMul} computes $z$\textit{+=}$x\odot y$, where $\odot$ denotes element-wise multiplication. In case of sparse vectors and an initially-empty output container, the primitives separate themselves in terms of the structure of the output vector, which is composed either of an intersection or union of the input structures.
\begin{itemize}
  \item \textbf{intersection (eWiseMul):} The primitive will compute only an element-wise multiplication for those positions where \emph{both} input containers have entries. This is since any missing entries are assumed to have value zero, and are therefore ignored under multiplication.
  \item \textbf{union (eWiseAdd):} The primitive will compute element-wise addition for those positions where \emph{any} of the input containers have entries. This is again because a missing entry in one of the containers is assumed to have a zero value, meaning the result of the addition simply equals the value of the entry present in the other container.
\end{itemize}

    \textbf{grb::mxv}, \textbf{grb::vxm} - Performs right- and left-handed matrix--vector multiplication; i.e., $u$\textit{+=}$Av$ and $u$\textit{+=}$vA$, respectively. More precisely, e.g., \texttt{grb::mxv} computes the standard linear algebraic operation $u_i = u_i + \sum_j A_{ij} v_j$. Different from \texttt{grb::set}, the \texttt{grb::mxv} is an in-place operation. If the intent is to compute $u=Av$ while $u$ is not empty, there are two solutions: 1) $u$ may cleared first (\texttt{grb::clear(u)}), or 2) $u$ may have all its values set to zero first (\texttt{grb::set(u, 0)}).\newline

    \textbf{grb::mxm} - Performs matrix--matrix multiplication; i.e., $C$\textit{+=}$AB$, or $C_{ij}=C_{ij}\textit{+=}\sum_{k}A_{ik}B_{kj}$. If, for a given $i,j$ the $i$th row of $A$ is empty or the $j$th column of $B$ is empty, no output will be appended to $C$-- that is, if $C_{ij}\notin C$, then after executing the primitive no such entry will have been added to $C$, meaning that the sparsity of $C$ is retained and the only fill-in to $C$ is due to non-zero contributions of $AB$. If in-place behaviour is not desired, $C$ must be cleared prior to calling this primitive\footnote{note that while setting $C$ to a dense matrix of all zeroes semantically also results in an out-of-place matrix--matrix multiplication, typically GraphBLAS applications should shy away from forming dense matrices due to their quadratic storage requirements}.\newline

    With all the above operations, the containers must have matching dimensions in the linear algebraic sense -- e.g., for $u=Av$, $u$ must have size equal to the number of rows in $A$ while $v$ must have size equal to the number of columns. If the sizes do not match, the related primitives will return \texttt{grb::MISMATCH}. Similarly, if for an output container the result cannot be stored due to insufficient capacity, \texttt{grb::ILLEGAL} will be returned. As with \texttt{grb:set}, furthermore, all above primitives may optionally take a mask as well as take a phase (resize or execute) as its last argument.

    There is one final caveat. In order to support graph algorithms, GraphBLAS provides on \emph{generalised} linear algebraic primitives -- not just numerical linear algebraic ones. This is explained further in the next subsection of this tutorial. For now, the only thing this necessitates is that each of the primitives explained above takes a mandatory \emph{semiring} argument. To perform standard numerical linear algebra, this last argument must be an instance of \texttt{grb::semirings::plusTimes< T >}, where $T$ is the domain over which we wish to perform numerical linear algebra (usually \texttt{double} or \texttt{std::complex< double >}). For example, to perform a sparse matrix--vector multiplication:
\begin{lstlisting}
auto plusTimes = grb::semirings::plusTimes< double >();
grb::RC rc = grb::clear( y );
rc = rc ? rc : grb::mxv(y, A, x, plusTimes);
\end{lstlisting}

\noindent \textbf{Exercise 8.} Copy the older \texttt{alp\_hw.cpp} to start with this exercise. Modify it to perform the following steps:
\begin{enumerate}
\item initialise a small matrix $A$ and vector $x$;
\item use \texttt{grb::set} and \texttt{grb::setElement} to assign values (see below for a suggestion);
\item perform a matrix--vector multiplication $y = Ax$ (using \texttt{grb::mxv} with the plus-times semiring);
\item compute a dot product $d = x^Tx$ (using \texttt{grb::dot} with the same semiring);
\item perform an element-wise multiplication $z = x \odot x$ (using \texttt{grb::eWiseMul} with the same semiring); and
\item print the results.
\end{enumerate}
One example $A, x$ could be:
\[
A = 
\begin{bmatrix}
0 & 1 & 2 \\
0 & 3 & 4 \\
5 & 6 & 0
\end{bmatrix},\quad
\mathbf{x} = \begin{bmatrix}1\\2\\3\end{bmatrix},
\]
for which, if the exercise is implemented OK, the output would be something like:
\begin{lstlisting} [language=bash, basicstyle=\ttfamily\small, showstringspaces=false]
Step 1: Constructing a 3x3 sparse matrix A.
Step 2: Creating vector x = [1, 2, 3]^T.
Step 3: Computing y = A·x under plus‐times semiring.
Step 4: Computing z = x ⊙ x (element‐wise multiply).
Step 5: Computing dot_val = xᵀ·x under plus‐times semiring.
x = [ 1, 2, 3 ]
y = A·x = [ 7, 18, 17 ]
z = x ⊙ x = [ 1, 4, 9 ]
dot(x,x) = 14
\end{lstlisting}

\noindent \textbf{Question.} If setting all $n$ elements of a size-$n$ vector via \texttt{grb::setElement}, what is its big-Omega (lower-bound asymptotic) complexity? Is the resulting complexity reasonable for a shared-memory parallel program?

\noindent \textbf{Bonus question.} If instead those $n$ elements are in some STL-type container for which we retrieve random access iterators, and ingest those elements using \texttt{grb::buildVectorUnique} and those random access iterators -- what would a good shared-memory parallel ALP/GraphBLAS implementation achieve in terms of complexity\footnote{if derived correctly, this bound is indeed achieved for the shared-memory parallel ALP/GraphBLAS backends that may be invoked and tested by passing \texttt{-b reference\_omp} and \texttt{-b nonblocking} to \texttt{grbcxx}}?

From these questions it follows that \texttt{grb::setElement} should only ever be used to set $\Theta(1)$ elements in some container and never asymptotically more than that. If a larger data set needs to be ingested into ALP containers, then in order to guarantee scalability, always rely on (parallel) iterators and the \texttt{grb::build*} primitives!

\subsection{Semirings and Algebraic Operations}

A key feature of GraphBLAS (and ALP) is that operations are defined over semirings rather than just the conventional arithmetic operations. A semiring consists of a pair of operations (an “addition” and a “multiplication”) along with their identity elements, which generalize the standard arithmetic (+ and $\times$). GraphBLAS allows using different semirings to, for example, perform computations like shortest paths or logical operations by substituting the plus or times operations with min, max, logical OR/AND, etc. In GraphBLAS, matrix multiplication is defined in terms of a semiring: the “add” operation is used to accumulate results, and the “multiply” operation is used when combining elements.
ALP lets you define and use custom \textbf{semirings} by specifying:


\begin{itemize}
  \item \textbf{A binary monoid:} an associative, commutative ``addition'' operation with an identity element. Examples:
  \begin{itemize}
    \item \texttt{(+}, 0\texttt{)} — the usual addition over numbers
    \item \texttt{(min}, $+\infty$\texttt{)} — useful for computing minima
  \end{itemize}
  
  \item \textbf{A binary multiplicative operator:} a second operation (not necessarily arithmetic multiplication), with its own identity element. Examples:
  \begin{itemize}
    \item \texttt{(*}, 1\texttt{)} — standard multiplication
    \item \texttt{(AND}, \texttt{true}\texttt{)} — logical semiring for Boolean algebra
  \end{itemize}
\end{itemize}

A semiring is a combination of a multiplicative operator and an additive monoid. Many common semirings are provided or can be constructed. For instance, the plus-times semiring uses standard addition as the accumulation (monoid) and multiplication as the combination operator – this yields ordinary linear algebra over real numbers. One can also define a \texttt{min-plus} semiring (useful for shortest path algorithms, where "addition" is min and "multiplication" is numeric addition). ALP’s design allows an “almost unlimited variety of operators and types” in semirings.

In code, ALP provides templates to construct these. For example, one can define:
\begin{lstlisting} [language=C++, basicstyle=\ttfamily\small, showstringspaces=false ]
using Add = grb::operators::add<double>;
using AddMonoid = grb::Monoid<Add, grb::identities::zero>;
using Mul = grb::operators::mul<double>;
using PlusTimes = grb::Semiring<Mul, AddMonoid>;
PlusTimes plusTimesSemiring;
\end{lstlisting}
Here we built the plus-times semiring for \texttt{double}: we use the provided addition operator and its identity (zero) to make a monoid, then combine it with the multiply operator to form a semiring. ALP comes with a library of predefined operator functors (in \texttt{grb::operators}) and identities (in \texttt{grb::identities}) for common types. You can also define custom functor structs if needed. In many cases, using the standard \texttt{plusTimesSemiring} (or simply passing operators/monoids directly to functions) is sufficient for basic algorithms.

\subsection{Primitive Operations (mxv, eWiseMul, dot, etc.)}

Using the above containers and semirings, ALP provides a set of primitive functions in the \texttt{grb} namespace to manipulate the data. These functions are free functions (not class methods) and typically take the output container as the first parameter (by reference), followed by input containers and an operator or semiring specification. The most important primitives include:

    \textbf{grb::set} – Assigns all elements of a container to a given value. For example, \texttt{grb::set(x, 1.0)} will set every entry of vector \texttt{x} to $1.0$ (making all indices present with value 1.0). This is useful for initialization (if called on an empty vector, it will insert all indices with that value). There is also \texttt{grb::setElement(container, value, index[, index2])} to set a single element: for a vector, you provide an index; for a matrix, a row and column. For example, \texttt{grb::setElement(y, 3.0, n/2)} sets $y_{n/2} = 3.0$.
\newline

    \textbf{grb::mxv} – Perform matrix-vector multiplication on a semiring. The call \texttt{grb::mxv(u, A, v, semiring)} computes $u = A \otimes v$ (where $\otimes$ denotes matrix-vector multiply under the given semiring). For the plus-times semiring, this corresponds to the usual linear algebra operation $u_i = \sum_j A_{ij} \times v_j$ (summing with + and multiplying with $\times$). The output vector \texttt{u} must be pre-allocated to the correct size (number of rows of $A$). By default, ALP’s \texttt{mxv} adds into the output vector (as if doing $u += A \times v$). If you want to overwrite \texttt{u} instead of accumulate, you may need to explicitly set \texttt{u} to the identity element (e.g. zero) beforehand or use descriptors (advanced options) – but for most use cases, initializing $u$ to 0 and then calling mxv is sufficient to compute $u = A x$. For example, \texttt{grb::mxv(y, A, x, plusTimesSemiring)} will compute $y_i = \sum_j A_{ij} x_j$ using standard arithmetic (assuming \texttt{y} was zeroed initially).
\newline

      \textbf{grb::dot} – Compute the dot product of two vectors. This is essentially a special case of a matrix-vector multiply or a reduce operation. ALP provides \texttt{grb::dot(result, u, v, semiring)} to compute a scalar result = $u^T \otimes v$ under a given semiring. For the standard plus-times semiring, \texttt{grb::dot(alpha, u, v, plusTimesSemiring)} will calculate $\alpha = \sum_i (u_i \times v_i)$ (i.e. the dot product of $u$ and $v$). If you use a different monoid or operator, you can compute other pairwise reductions (for example, using a \texttt{min} monoid with logical multiplication could compute something like an “AND over all i” if that were needed). In most cases, you'll use dot with the default arithmetic semiring for inner products. The output \texttt{alpha} is a scalar (primitive type) passed by reference, which will be set to the resulting value.
\newline

      \textbf{grb::apply} – Apply a unary operator or indexed operation to each element of a container. The function \texttt{grb::apply(z, x, op)} applies a given unary functor \texttt{op} to each element of vector (or matrix) \texttt{x}, writing the result into \texttt{z}. For example, if \texttt{op} is a functor that squares a number, \texttt{grb::apply(z, x, op)} would produce $z_i = \textit{op}(x_i)$ for all stored elements of $x$. There are also forms of \texttt{apply} that use a binary operator with a scalar, effectively applying an affine operation. For instance, ALP could support \texttt{grb::apply(z, x, grb::operators::add<double>(), 5.0)} to add 5 to each element of $x$ (if such an overload exists – conceptually, this would treat it as $z_i = x_i + 5$ for all $i$). In general, \texttt{apply} is like a map operation over all elements (it does not change the sparsity pattern — if $x$ has an element at $i$, $z$ will have an element at $i$ after apply, unless filtered by a mask).
    In summary, to do element-wise multiplication (intersection), you might call:
    \texttt{grb::eWiseApply(z, x, y, grb::operators::mul<double>())},
    which computes $z_i = x_i \times y_i$ for each $i$ that has both $x_i$ and $y_i$. To do an element-wise addition (union), you could call
    \texttt{grb::eWiseApply(z, x, y, addMonoid)}, where \texttt{addMonoid} encapsulates “+” with identity 0, resulting in $z_i = x_i + y_i$ for all indices that exist in either $x$ or $y$. (If an index is missing in one operand, that operand contributes 0.)

\paragraph{API usage notes:} All the above operations require that output parameters be passed by reference, since they are modified in place (e.g., \texttt{y} in \texttt{grb::mxv(y, A, x, ...)} is updated with the result). Input objects are typically passed by const-reference. You should ensure that the output container is allocated with the correct size beforehand – ALP will not automatically resize vectors or matrices on operation calls if dimensions mismatch. If dimensions do not agree (e.g., you try to multiply an $m\times n$ matrix with a vector of length not $n$), the function will return an error code to indicate the misuse. In fact, most ALP primitives return a status code of type \texttt{grb::RC} (with \texttt{grb::SUCCESS} indicating success). For clarity, our code examples will omit explicit error handling, but in a real program you may check the returned code of each operation.

In the next section, we will put these concepts together in a concrete example.


\section{Solution to Exercise 8}\label{sec:simple_example}

Below a sample solution code for this example, with commentary:\footnote{Maybe better to move this to the solutions directory on the tutorial repository? (TODO)}

\begin{lstlisting}[ language=C++, basicstyle=\ttfamily\small, caption={Example program using ALP/GraphBLAS primitives in C++}, label=lst:example, showstringspaces=false]

/*
 * example.cpp - Corrected minimal ALP (GraphBLAS) example.
 *
 * To compile (using the reference OpenMP backend):
 *    grbcxx -b reference_omp example.cpp -o example
 *
 * To run:
 *    grbrun ./example
 */

#include <cstdio>
#include <iostream>
#include <vector>
#include <utility>   // for std::pair
#include <array>

#include <graphblas.hpp>

using namespace grb;

// Indices and values for our sparse 3x3 matrix A:
//
//      A = [ 1   0   2 ]
//          [ 0   3   4 ]
//          [ 5   6   0 ]
//
// We store the nonzero entries via buildMatrixUnique.
static const size_t Iidx[6]    = { 0, 0, 1, 1, 2, 2 };  // row indices
static const size_t Jidx[6]    = { 0, 2, 1, 2, 0, 1 };  // column indices
static const double Avalues[6] = { 1.0, 2.0, 3.0, 4.0, 5.0, 6.0 };

int main( int argc, char **argv ) {
    (void)argc;
    (void)argv;
    std::printf("example (ALP/GraphBLAS) corrected API usage\n\n");

    //------------------------------
    // 1) Create a 3x3 sparse matrix A
    //------------------------------
    std::printf("Step 1: Constructing a 3x3 sparse matrix A.\n");
    Matrix<double> A(3, 3);
    // Reserve space for 6 nonzeros
    resize(A, 6);
    // Populate A from (Iidx,Jidx,Avalues)
    buildMatrixUnique(
        A,
        &(Iidx[0]),
        &(Jidx[0]),
        Avalues,
        /* nvals = */ 6,
        SEQUENTIAL
    );

    //------------------------------
    // 2) Create a 3-element vector x and initialize x = [1, 2, 3]^T
    //------------------------------
    std::printf("Step 2: Creating vector x = [1, 2, 3]^T.\n");
    Vector<double> x(3);
    set<descriptors::no_operation>(x, 0.0);           // zero-out
    setElement<descriptors::no_operation>(x, 1.0, 0); // x(0) = 1.0
    setElement<descriptors::no_operation>(x, 2.0, 1); // x(1) = 2.0
    setElement<descriptors::no_operation>(x, 3.0, 2); // x(2) = 3.0

    //------------------------------
    // 3) Create two result vectors y and z (dimension 3)
    //------------------------------
    Vector<double> y(3), z(3);
    set<descriptors::no_operation>(y, 0.0);
    set<descriptors::no_operation>(z, 0.0);

    //------------------------------
    // 4) Use the built-in “plusTimes” semiring alias
    //      (add = plus, multiply = times, id‐add = 0.0, id-mul = 1.0)
    //------------------------------
    auto plusTimes = grb::semirings::plusTimes<double>();

    //------------------------------
    // 5) Compute y = A·x  (matrix‐vector multiply under plus‐times)
    //------------------------------
    std::printf("Step 3: Computing y = A·x under plus‐times semiring.\n");
    {
        RC rc = mxv<descriptors::no_operation>(y, A, x, plusTimes);
        if(rc != SUCCESS) {
            std::cerr << "Error: mxv(y,A,x) failed with code " << toString(rc) << "\n";
            return (int)rc;
        }
    }

    //------------------------------
    // 6) Compute z = x ⊙ x  (element‐wise multiply) via eWiseApply with mul
    //------------------------------
    std::printf("Step 4: Computing z = x ⊙ x (element‐wise multiply).\n");
    {
        RC rc = eWiseApply<descriptors::no_operation>(
            z, x, x,
            grb::operators::mul<double>()   // plain multiplication ⊙
        );
        if(rc != SUCCESS) {
            std::cerr << "Error: eWiseApply(z,x,x,mul) failed with code " << toString(rc) << "\n";
            return (int)rc;
        }
    }
    
    //------------------------------
    // 7) Compute dot_val = xᵀ·x  (dot‐product under plus‐times semiring)
    //------------------------------
    std::printf("Step 5: Computing dot_val = xᵀ·x under plus‐times semiring.\n");
    double dot_val = 0.0;
    {
        RC rc = dot<descriptors::no_operation>(dot_val, x, x, plusTimes);
        if(rc != SUCCESS) {
            std::cerr << "Error: dot(x,x) failed with code " << toString(rc) << "\n";
            return (int)rc;
        }
    }

    //------------------------------
    // 8) Print x, y, z, and dot_val
    //    We reconstruct each full 3 - vector by filling an std::array<3,double>
    //------------------------------
    auto printVector = [&](const Vector<double> &v, const std::string &name) {
        // Initialize all entries to zero
        std::array<double,3> arr = { 0.0, 0.0, 0.0 };
        // Overwrite stored (nonzero) entries
        for(const auto &pair : v) {
            // pair.first = index, pair.second = value
            arr[pair.first] = pair.second;
        }
        // Print
        std::printf("%s = [ ", name.c_str());
        for(size_t i = 0; i < 3; ++i) {
            std::printf("%g", arr[i]);
            if(i + 1 < 3) std::printf(", ");
        }
        std::printf(" ]\n");
    };

    std::printf("\n-- Results --\n");
    printVector(x, "x");
    printVector(y, "y = A·x");
    printVector(z, "z = x ⊙ x");
    std::printf("dot(x,x) = %g\n\n", dot_val);

    return EXIT_SUCCESS;
}
\end{lstlisting}

In this program, we manually set up a $3\times3$ matrix $A$:
\[
A = \begin{pmatrix}
1 & 0 & 2 \\
0 & 3 & 4 \\
5 & 6 & 0
\end{pmatrix},
\]


and a vector $x = [1,2,3]^T$. The code multiplies $A$ by $x$, producing $y = A \times x$. Given the above $A$ and $x$, the result should be:

\[
y = \begin{pmatrix}
7 \\
18 \\
17
\end{pmatrix},
\]

because
$y_0 = 1\cdot 1 + 0\cdot 2 + 2\cdot 3 = 7$,
$y_1 = 0\cdot1 + 3\cdot2 + 4\cdot3 = 18$,
$y_2 = 5\cdot1 + 6\cdot2 + 0\cdot3 = 17$.

We also compute the dot product $x \cdot x = 1^2 + 2^2 + 3^2 = 14$ and the element-wise product $z = x * x = [1,4,9]^T$. The program then extracts the results with \texttt{grb::extractElement} (to get values from the containers) and prints them. Running this program would produce output similar to:

\begin{verbatim}
y = [7, 18, 17]
x . x = 14
z (element-wise product of x with x) = [1, 4, 9]
\end{verbatim}

This confirms that our ALP operations worked as expected. The code demonstrates setting values, performing an \texttt{mxv} multiplication, an element-wise multiply, and a dot product, covering several fundamental ALP/GraphBLAS operations.


\section{Makefile and CMake Instructions}\label{sec:build_instructions}

Finally, we provide guidance on compiling and running the above example in your own development environment. If you followed the installation steps and used \texttt{grbcxx}, compilation is straightforward. Here we outline two approaches: using the ALP wrapper scripts, and integrating ALP manually via a build system.

\subsection*{Using the ALP compiler wrapper}

The simplest way to compile your ALP-based program is to use the provided wrapper. After sourcing the ALP environment (setenv script), the commands \texttt{grbcxx} and \texttt{grbrun} are available in your PATH. You can compile the example above by saving it (e.g. as \texttt{example.cpp}) and running:
\begin{lstlisting}[language=bash]
$ grbcxx example.cpp -o example
\end{lstlisting}
This will automatically invoke \texttt{g++} with the correct include directories and link against the ALP library and its required libraries (NUMA, pthread, etc.). By default, it uses the sequential backend. To enable parallel execution with OpenMP, add the flag \texttt{-b reference\_omp} (for shared-memory parallelism). For instance:


\begin{lstlisting}[language=bash]
$ grbcxx -b reference_omp example.cpp -o example
\end{lstlisting}
After compilation, run the program with:
\begin{lstlisting}[language=bash]
$ grbrun ./example
\end{lstlisting}

(You can also run \texttt{./example} directly for a non-distributed run; \texttt{grbrun} is mainly needed for orchestrating distributed runs or setting up the execution environment.)

\subsection*{Using a custom build (Make/CMake)}

If you prefer to compile without the wrapper (for integration into an existing project or custom build system), you need to instruct your compiler to include ALP's headers and link against the ALP library and dependencies. The ALP installation (at the chosen \texttt{--prefix}) provides an include directory and a library directory.

For example, if ALP is installed in \texttt{../install} as above, you could compile the program manually with:
\begin{lstlisting}[language=bash]
$ g++ -std=c++17 example.cpp
-I../install/include -L../install/lib
-lgraphblas -lnuma -lpthread -lm -fopenmp -o example
\end{lstlisting}
Here we specify the include path for ALP headers and link with the ALP GraphBLAS library (assumed to be named \texttt{libgraphblas}) as well as libnuma, libpthread, libm (math), and OpenMP (the \texttt{-fopenmp} flag). These additional libraries are required by ALP (as noted in the install documentation). Using this command (or a corresponding Makefile rule) will produce the executable \texttt{example}.

If you are using CMake for your own project, you can integrate ALP as follows. There may not be an official CMake package for ALP, but you can use \texttt{find\_library} or hard-code the path. For instance, in your \texttt{CMakeLists.txt}:

\begin{lstlisting}[caption={Example CMakeLists.txt for an ALP project}]
    cmake_minimum_required(VERSION 3.13)
    project(ALPExample CXX)
    find_package(OpenMP REQUIRED) # find OpenMP for -fopenmp
    Set ALP install paths (adjust ../install to your prefix)
    
    include_directories("../install/include")
    link_directories("../install/lib")
    
    add_executable(example example.cpp)
    target_link_libraries(example PRIVATE
        graphblas # ALP GraphBLAS library
        OpenMP::OpenMP_CXX # OpenMP (threading support)
        pthread numa m # Pthreads, NUMA, math libraries
    )
\end{lstlisting}

This will ensure the compiler knows where to find \texttt{graphblas.hpp} and links the required libraries. After configuring with CMake and building (via \texttt{make}), you can run the program as usual.

\vspace{1ex}

Note: If ALP provides a CMake package file, you could use \texttt{find\_package} for ALP, but at the time of writing, linking manually as above is the general approach. Always make sure the library names and paths correspond to your installation. The ALP documentation mentions that programs should link with \texttt{-lnuma}, \texttt{-lpthread}, \texttt{-lm}, and OpenMP flags in addition to the ALP library.

\bigskip

This tutorial has introduced the fundamentals of using ALP/GraphBLAS in C++ on Linux, from installation to running a basic example. With ALP set up, you can explore more complex graph algorithms and linear algebra operations, confident that the library will handle parallelization and optimization under the hood. Happy coding!
