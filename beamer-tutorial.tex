\documentclass[aspectratio=169]{beamer}
\usetheme{Madrid}
\usecolortheme{seahorse}
\setbeamertemplate{navigation symbols}{}
\setbeamertemplate{footline}[frame number]

\usepackage[utf8]{inputenc}
\usepackage[T1]{fontenc}
\usepackage{lmodern}
\usepackage{graphicx}
\usepackage{amsmath,amssymb,mathtools}
\usepackage{hyperref}
\usepackage{xcolor}
\usepackage{listings}

\newcommand{\PA}[1]{{\textcolor{cyan}{[{\bfseries PA:} #1]}}}
\newcommand{\DeJ}[1]{{\textcolor{magenta}{[{\bfseries DJ:} #1]}}} % DJ was taken!
\newcommand{\AJ}[1]{{\textcolor{green}{[{\bfseries AJ:} #1]}}}
\newcommand{\GG}[1]{{\textcolor{blue}{[{\bfseries GG:} #1]}}}

% Map common Unicode symbols to LaTeX math (outside listings)
\usepackage{newunicodechar}
\newunicodechar{·}{\ensuremath{\cdot}}
\newunicodechar{⊙}{\ensuremath{\odot}}
\newunicodechar{ᵀ}{\ensuremath{^{\top}}}
\newunicodechar{×}{\ensuremath{\times}}
\newunicodechar{–}{-}
\newunicodechar{—}{-}

% Listing styles (Unicode-safe for later use)
\definecolor{terminalback}{rgb}{0.1,0.1,0.1}
\definecolor{terminaltext}{rgb}{0.9,0.9,0.9}
\lstdefinestyle{terminal}{
    backgroundcolor=\color{terminalback},
    basicstyle=\ttfamily\small\color{terminaltext},
    frame=single, rulecolor=\color{white}, breaklines=true,
    captionpos=b, showstringspaces=false,
    upquote=true, columns=fullflexible,
    literate={·}{{$\cdot$}}1 {⊙}{{$\odot$}}1 {ᵀ}{{$^\top$}}1 {×}{{$\times$}}1 {‐}{{-}}1 {–}{{-}}1 {—}{{-}}1
}
\lstdefinestyle{code}{
    backgroundcolor=\color[rgb]{0.95,0.95,0.95},
    basicstyle=\ttfamily\small, frame=single, breaklines=true,
    rulecolor=\color{black}, numbers=none, keepspaces=true, captionpos=b, tabsize=2,
    language=C++,
    literate={·}{{$\cdot$}}1 {⊙}{{$\odot$}}1 {ᵀ}{{$^\top$}}1 {×}{{$\times$}}1 {‐}{{-}}1 {“}{{"`}}1 {”}{{'"}}1
}

\title{Algebraic Programming (ALP) Tutorial}
\subtitle{From HPC to GraphBLAS and Transition Paths}
\author{ALP Team}
\date{\today}

\begin{document}
\frame{\titlepage}

% =========================
% Monday 10 Nov
% =========================
\section{Monday 10 Nov}

\subsection{1) Introduction to GraphBLAS and ALP}
\begin{frame}{Introduction to GraphBLAS and ALP}
\begin{itemize}
    \item Goals of the tutorial
    \item What is ALP and GraphBLAS
    \item Overview of containers, semirings, and primitives
\end{itemize}
% Owner: DJ
% Source (introduction.tex):
% \section{Introduction}\label{sec:introduction}
% ALgebraic Programming, or ALP for short, encompasses software technologies...
\end{frame}

\subsection{2) Installation on Linux, demos}
\begin{frame}[fragile]{ALP Installation}
\framesubtitle{Prerequisites (skip if already installed)}
\begin{itemize}
    \item Requirements:
    \begin{itemize}
        \item C++11 compiler, OpenMP, CMake (>=3.13), libNUMA, pthreads
    \end{itemize}
    \item Debian/Ubuntu:
\begin{lstlisting}[style=terminal, language=bash]
sudo apt-get install build-essential libnuma-dev libpthread-stubs0-dev cmake
\end{lstlisting}
    \item Red Hat/CentOS:
\begin{lstlisting}[style=terminal, language=bash]
dnf group install "Development Tools"
dnf install numactl-devel cmake
\end{lstlisting}
\end{itemize}
% Owner: PA
% Source (ALP_Tutorial.tex - Installation on Linux, step 1)
\end{frame}

\begin{frame}[fragile]{ALP Installation}
\framesubtitle{Obtain and Build ALP}
\begin{enumerate}
    \item Clone \textbf{ALP} from the official GitHub repository:
\begin{lstlisting}[style=terminal, language=bash]
git clone https://github.com/Algebraic-Programming/ALP.git
\end{lstlisting}
    \item Build and install \textbf{ALP} with default configuration settings:
\begin{lstlisting}[style=terminal, language=bash]
cd ALP && mkdir build && cd build
../bootstrap.sh --prefix=../install
make -j
make install
\end{lstlisting}
\end{enumerate}
% Owner: PA
% Source (ALP_Tutorial.tex - Installation on Linux, steps 2-3)
\end{frame}

\begin{frame}[fragile]{ALP Installation}
\framesubtitle{Setup Environment and Test}
\begin{enumerate}
    \item \textbf{Activate ALP environment}:
\begin{lstlisting}[style=terminal, language=bash]
source ../install/bin/setenv
\end{lstlisting}
    \item \textbf{Compile example}:
\begin{lstlisting}[style=terminal, language=bash]
grbcxx ../examples/sp.cpp -o sp_example
\end{lstlisting}
    \item \textbf{Run}:
\begin{lstlisting}[style=terminal, language=bash]
grbrun ./sp_example
\end{lstlisting}
\end{enumerate}
% Owner: PA
% Source (ALP_Tutorial.tex - Installation on Linux, steps 4-6)
\end{frame}

\begin{frame}[fragile]{Hands-on preparation: Setting up scripts}
    \framesubtitle{ALP/GraphBLAS Overview}
    \begin{itemize}
        \item You can find code skeletons for the tutorial in: \texttt{/some/path/to/alp/tutorial/scripts}
        \item You should copy the scripts to your home directory: 
\begin{lstlisting}[style=terminal, language=bash]
cp -r /some/path/to/alp/tutorial/scripts ~/alp_tutorial_scripts
cd ~/alp_tutorial_scripts && ls -l
\end{lstlisting}
        \item You can use any editor of your preference to edit these scripts (e.g. nano, vim, gedit)
        \item Run these commands to test your setup:
\begin{lstlisting}[style=terminal, language=bash]
grbcxx alp_hw.cpp
grbrun ./a.out
\end{lstlisting}
\item \textbf{Expected output}:
\begin{lstlisting}[style=terminal, language=bash]
Info: grb::init (reference) called.
Hello from ./a.out
Info: grb::finalize (reference) called.
\end{lstlisting}
    \end{itemize}
\end{frame}

\begin{frame}[fragile]{Hands-on preparation: What did we just run?}
    \framesubtitle{Hello World in ALP/GraphBLAS}
    \vspace{-0.6em}
    \begin{lstlisting}[style=code, language=C++, basicstyle=\ttfamily\scriptsize]
#include <cstddef>
#include <cstring>
#include <graphblas.hpp>
#include <assert.h>

constexpr size_t max_fn_size = 255;
typedef char Filename[ max_fn_size ];

void hello_world( const Filename &in, int &out ) {
    std::cout << "Hello from " << in << std::endl;
    out = 0;
}

int main( int argc, char ** argv ) {
    Filename fn;
    std::strncpy( fn, argv[0], max_fn_size );
    int error_code = 100;
    
    grb::Launcher< grb::AUTOMATIC > launcher;
    launcher.exec( &hello_world, fn, error_code, true );
    return error_code;
}
        \end{lstlisting}
        % Owner: PA
        % Source (ALP_Tutorial.tex - ALP/GraphBLAS, Hello World)
\end{frame}

\begin{frame}{Hands-on preparation: What did we just run?}
\framesubtitle{ALP/GraphBLAS Overview \PA{Some of this info might/should be at the intro. Adjust accordingly.}}
\begin{itemize}
    \item Pure C++ (developed similarly to the GraphBLAS C specification)
    \item Exposes a GraphBLAS interface with 3 categories (part of the \texttt{grb} namespace):
    \begin{itemize}
        \item Algebraic containers (vectors, matrices, etc.)
        \item Algebraic structures (binary operators, semirings, etc.)
        \item Algebraic operations (take containers and structures as arguments)
    \end{itemize}
    \item \textbf{\texttt{grb::Launcher}}:
    \begin{itemize}
        \item Wraps calls to ALP programs
        \item Adapts to run-time conditions (e.g., distributed execution) \PA{Check what this means}
    \end{itemize}
    \item All ALP programs: input (\href{https://www.geeksforgeeks.org/cpp/pod-type-in-cpp/}{\textcolor{blue}{POD}}) $\rightarrow$ output (POD)
    \begin{itemize}
        \item \textbf{\texttt{hello\_world} example}:
        \begin{itemize}
            \item Function, filename passed as input (POD type)
            \item Returns zero error\_code as output (POD type)
        \end{itemize}
    \end{itemize}
\end{itemize}
\vfill
\colorbox{gray!20}{
\begin{minipage}{0.95\textwidth}
\small
\textbf{For more info:} ALP Documentation: \url{http://albert-jan.yzelman.net/alp/user/}
\end{minipage}}
% Owner: PA
% Source (ALP_Tutorial.tex - ALP/GraphBLAS, Hello World explanation)
\end{frame}

\subsection{3) Hands-on: containers, I/O, copying, masking, standard matrices (3.1–3.3)}
\begin{frame}[fragile]{Hands-on: Containers and I/O (3.1)}
\begin{itemize}
    \item grb::Vector, grb::Matrix basics
    \item Capacity, nnz, sizes
    \item Iterators and output
\end{itemize}
% Owner: PA
% Source (ALP_Tutorial.tex - ALP/GraphBLAS Containers):
% The primary ALP/GraphBLAS container types are \texttt{grb::Vector<T>} and \texttt{grb::Matrix<T>}.
% \begin{lstlisting}
% grb::Vector<double> x(100000), y(150000);
% grb::Matrix<void> A(150000, 100000);
% \end{lstlisting}
% ... capacity, size, nnz, default capacities ...
\end{frame}

\begin{frame}[fragile]{Hands-on: Basic Container I/O (3.2)}
\begin{itemize}
    \item set, setElement, clear
    \item Iterating and printing
    \item Reading MatrixMarket via parser
\end{itemize}
% Owner: PA
% Source (ALP_Tutorial.tex - Basic Container I/O):
% \texttt{grb::set(vector,scalar)} sets all elements; \texttt{grb::setElement} sets one.
% \begin{lstlisting}
% for( const auto &pair : y ) {
%   std::cout << "y[ " << pair.first << " ] = " << pair.second << "\n";
% }
% \end{lstlisting}
% Parser:
% \begin{lstlisting}
% #include <graphblas/utils/parser.hpp>
% grb::utils::MatrixFileReader<double> parser( in, true );
% grb::buildMatrixUnique(A, parser.begin(SEQUENTIAL), parser.end(SEQUENTIAL), SEQUENTIAL);
% \end{lstlisting}
\end{frame}

\begin{frame}[fragile]{Hands-on: Copying, Masking, and Standard Matrices (3.3)}
\begin{itemize}
    \item Copy with resize/execute phases
    \item Masks controlling writes
    \item Factory: identity, eye, diag
\end{itemize}
% Owner: PA
% Source (ALP_Tutorial.tex - Copying, Masking, and Standard Matrices):
% \begin{lstlisting}
% grb::Matrix<double> B(497, 497);
% grb::RC rc = grb::set( x, y );
% rc = rc ? rc : grb::set( B, A, grb::RESIZE );
% rc = rc ? rc : grb::set( B, A, grb::EXECUTE );
% \end{lstlisting}
% Mask example:
% \begin{lstlisting}
% grb::RC rc = grb::set( x, y, false );
% \end{lstlisting}
% Factory:
% \begin{lstlisting}
% #include <graphblas/algorithms/matrix_factory.hpp>
% const auto I = grb::algorithms::matrices<double>::identity(n);
% \end{lstlisting}
\end{frame}

\subsection{4) Introduction to primitives (3.4)}
\begin{frame}{Primitives overview (3.4)}
\begin{itemize}
    \item Level-1: foldl/foldr, dot, eWiseAdd/Mul
    \item Level-2: mxv, vxm
    \item Level-3: mxm
\end{itemize}
% Owner: AJ
% Source (ALP_Tutorial.tex - Numerical Linear Algebra):
% Primitives: \texttt{grb::foldl}, \texttt{grb::dot}, \texttt{grb::eWiseAdd}, \texttt{grb::eWiseMul},
% \texttt{grb::mxv}, \texttt{grb::vxm}, \texttt{grb::mxm}.
% \begin{lstlisting}
% auto plusTimes = grb::semirings::plusTimes<double>();
% grb::mxv(y, A, x, plusTimes);
% \end{lstlisting}
\end{frame}

\subsection{5) Exercise 8}
\begin{frame}[fragile]{Exercise 8}
\begin{itemize}
    \item Build small A and x
    \item Compute y=Ax, z=x⊙y, d=xᵀx
    \item Print results
\end{itemize}
% Owner: DJ
% Source (ALP_Tutorial.tex - Exercise 8):
% One example:
% A = [ [0,1,2], [0,3,4], [5,6,0] ], x = [1,2,3]^T
% Expected:
% \begin{lstlisting}[language=bash]
% x = [ 1, 2, 3 ]
% y = A·x = [ 7, 18, 17 ]
% z = x ⊙ y = [ 7, 36, 51 ]
% dot(x,x) = 14
% \end{lstlisting}
\end{frame}

\subsection{6) Closure, transition paths, python mxv examples}
\begin{frame}{Transition paths and Python mxv}
\begin{itemize}
    \item ALP transition paths overview
    \item Re-linking existing codes
    \item Python examples
\end{itemize}
% Owner: DJ & AJ
% Source (ALP_Transition_Path_Tutorial.tex - Intro):
% ALP ... transition path APIs ... drop-in replacements ... re-link your program against ALP's libraries ...
\end{frame}

\subsection{7) Ising Machine, new solvers}
\begin{frame}{Ising Machine, new solvers}
\begin{itemize}
    \item Placeholder for Ising Machine content
    \item New solver directions
\end{itemize}
% Owner: DJ
\end{frame}

% =========================
% Tuesday 11 Nov
% =========================
\section{Tuesday 11 Nov}
\subsection{8) Finish Exercise 8}
\begin{frame}{Finish Exercise 8}
\begin{itemize}
    \item Review results and pitfalls
    \item Alternative builds and descriptors
\end{itemize}
% Owner: DJ
\end{frame}

\subsection{9) Semirings, monoids, operators (add exercise: shortest path)}
\begin{frame}[fragile]{Semirings, monoids, operators}
\begin{itemize}
    \item Defining and using semirings
    \item Built-ins: plusTimes, minPlus, boolean
    \item Exercise: shortest path (min-plus)
\end{itemize}
% Owner: AJ 
% TODO: add exercise "shortest path"
% Source (ALP_Tutorial.tex - Semirings and Algebraic Operations):
% A semiring consists of a pair of operations ...
% \begin{lstlisting}[language=C++]
% using Add = grb::operators::add<double>;
% using AddMonoid = grb::Monoid<Add, grb::identities::zero>;
% using Mul = grb::operators::mul<double>;
% using PlusTimes = grb::Semiring<Mul, AddMonoid>;
% \end{lstlisting}
\end{frame}

\subsection{10) Solvers, transition path, Python API, ABI}
\begin{frame}{Solvers and transition path}
\begin{itemize}
    \item Sparse CG solver API
    \item Preconditioners
    \item ABI and Python API notes
\end{itemize}
% Owner: DJ
% Source (ALP_Transition_Path_Tutorial.tex - API):
% sparse_cg_init / set_preconditioner / solve / destroy — CRS inputs; non-blocking engine, synchronous API.
\end{frame}

\subsection{11) Hands-on: Section 8; CG example; Python example}
\begin{frame}[fragile]{Hands-on: CG example}
\begin{itemize}
    \item Build CRS matrix
    \item Call CG init/solve/destroy
    \item Validate solution
\end{itemize}
% Owner: DJ
% Source (ALP_Transition_Path_Tutorial.tex - Example):
% \begin{lstlisting}[language=C++]
% // A (CRS), b, x; sparse_cg_init_dii(&handle, n, A_vals, A_cols, A_offs);
% // sparse_cg_solve_dii(handle, x, b); sparse_cg_destroy_dii(handle);
% \end{lstlisting}
\end{frame}

\subsection{12) Stretch goal: SPMD and bsp1d backend}
\begin{frame}{SPMD and bsp1d backend (stretch)}
\begin{itemize}
    \item Backend concepts
    \item Launching distributed runs
\end{itemize}
% Owner: AJ  % TODO: install lpf
\end{frame}

\subsection{13) Performance model, high level intro and results}
\begin{frame}{Performance model}
\begin{itemize}
    \item BSP/HBSP overview
    \item Cost components and examples
\end{itemize}
% Owner: PA/
\end{frame}

\subsection{14) SPMD execution, Replica exchange}
\begin{frame}{SPMD execution, Replica exchange}
\begin{itemize}
    \item Concepts and API sketch
    \item Example workflow
\end{itemize}
% Owner: GG
\end{frame}

% =========================
% Wednesday 12 Nov
% =========================
\section{Wednesday 12 Nov}

\subsection{15) WIP overview: dense backend, tensor, stencil, EM simulations}
\begin{frame}{WIP overview}
\begin{itemize}
    \item Dense backend status
    \item Tensor/stencil/EM simulation directions
\end{itemize}
% Owner: DJ
\end{frame}

\subsection{16) Deep technical? Future work (autodiff)}
\begin{frame}{Future work (autodiff)}
\begin{itemize}
    \item Autodiff integration ideas
    \item Open problems and roadmap
\end{itemize}
% Owner: DJ
\end{frame}

\end{document}